\chapter{Implementation}



\section{Overview}



\section{AWS EC2}
There were two AWS instances deployed for the implementation of the project. Both were hosted in the Amazon US East (Ohio) availability region with Ubuntu 16.04 LTS operating systems, on servers with different hardware resources.

\begin{itemize}
	
	\item Management Instance
	Since the management instance was required to host a number of visualisation tools, there were certain minimum hardware resources required to be able to run these. Due to cost considerations, a minimum set of resources were allocated: a server with 4GB of RAM and 2 virtual CPUs was provisioned. 
	
	\item Honeypot Instance
	
	In order to be able to host a number of docker containers, the hardware requirements of the AWS instance were greater than those provided with the 12-month free-tier offer. Instead, a machine with 16GB RAM and 4 virtual CPUs was provisioned in order to allow a sufficient number of docker containers to be hosted on the device. 
	
	
\end{itemize}


\includecode{Short Caption}{Lengthy caption explaining the code to the reader}{snippet.py}


\section{PSAD}

\section{Docker}

	\subsection{Cowrie Containers}

	\subsection{Router Container}
	
	\subsection{The Docker Honeynet}
	- Define bridge network (dmz)
	- Specify IP addresses for everyone on the network
	
\section{Logging and Visualisation}

The Elasticsearch-Logstash-Kibana (ELK) stack is widely used as a means of log analysis, aggregation and visualisation. This meant that there were plentiful resources available from which to learn how to configure these tools to work together.

Some complementary tools were also used in order to facilitate the use of the ELK stack for analysis of logs from a remote machine.

	\subsection{Elasticsearch}
	\subsection{Kibana}
	\subsection{Nginx}
	\subsection{Logstash}
	\subsection{Filebeat}
		\subsubsection{Crontab}
		
		A cron job was configured to run every 60 seconds on the honeypot instance, shipping the logs from a defined location on the honeypot instance to the \textit{/var/logs/} directory. 

\section{Bash Scripting}

Bash scripts were used in order to make the entire setup maximally reproducable. \textbf{These can be found in \underline{Appendix A}}
