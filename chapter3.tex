\chapter{Background}

Starting with a general description of the issue; then drilling down into the details of the issue and how it has been covered in the literature. For a skeleton at the beginning of the writing, this text should be replaced by a general short description, so that you know what you want to discuss and can review the sequence of the discussion.

\section{Challenges in Modern Security}

\subsection{Cyber-Incident Monitoring}




\section{Problem Formulation}

As discussed at length, the continuously advancing nature of security threats in response to technological developments and innovations means that the research and documentation of techniques, tools and behaviours in cybersecurity quickly become outdated. One thing is very clear: That there is an urgent need for re-evaluation of the cybersecurity defence strategies of critical service infrastructures in order to protect the wider global community. Ultimately, it is this consideration that is at the heart of formulating the research problem.

Taking the compromise of the NHS infrastructure due to the WannaCry infection in 2017 gives a great basis for addressing many issues seen across critical service infrastructures. Being such a high-impact incident, this event in particular has emphasised the severe consequences of such an attack on a critical service.

	\subsection{Shortcomings in Critical Service Infrastructures}
	
	At present, IT systems in many organisations are comparable to a “black-box system”, a term commonly used in engineering where there is a lack of feedback mechanisms present in the system in question. (\textit{Insert reference to definition of this concept}) Such a system gives no indication of what is happening internally until it starts to behave unexpectedly, making it difficult to diagnose problems and almost impossible to resolve them. In this regard, it is necessary to recognise that although the use of traditional intrusion detection systems such as firewalls hold an important place in securing IT systems, they have limited diagnostic capabilities.
	
	The compromise of the NHS systems illustrates a number of substantial shortcomings in their cybersecurity defence strategies up to 12th May 2017 when the WannaCry attacks occurred, which were identified in an audit report [6] by the UK National Audit Office. 
	
	\begin{enumerate}
		\item There were no established protocols in place in the event that such a high-impact cyber incident would occur. As a consequence of much misdirected communication, it took approximately half a day for key decision makers and first responders to be informed. 
		
		\item There was no incident monitoring in place to alert security first-responders to any unusual behaviour in the NHS systems: Evident from the fact that it took over half a day to reach those who could take remedial action. 
		
		\item There was no centralisation of system data, meaning that security experts had to physically attend the affected sites in order to collect crucial data about the attacks. This further prolonged the period for which these systems remained inoperable. 
	
		\item Recommendations regarding system updates for the NHS IT systems were not followed, and there was no system in place to ensure that the actions had been taken. Outdated and unpatched operating systems were a major contributor to the success of the compromise, which was entirely based on a Windows exploit.5
	
	\end{enumerate}
	
	Ultimately, the lack of monitoring of the NHS systems meant that when the attacks occurred, the systems at the affected sites remained offline for an extended period of time, causing major issues across the entire healthcare system in the UK. Using honeypots as part of  intrusion detection and active network defence could provide a solution to this. The ability of honeypots to produce cyber-incident alerts with a very low number of false positives, as well as capturing detailed information about interactions with attackers, makes a honeypot-based system a potential solution to the issues encountered by the NHS in May 2017.

	\subsection{The Potential of Honeypot-Driven Incident Montoring}
	A production-ready system leveraging off the ability of honeypots to proactively detect and learn from attacks would provide a great deal of extra knowledge to system administrators about the flaws in their IT infrastructure. Such a system must be low-cost, low-maintenance and highly effective: A cyber-incident monitoring framework that can be easily deployed in an enterprise-scale network. It will need to appeal to organisations in terms of: (i) Cost, (ii) effectiveness in securing their network, (iii) reliability, and (iv) maintenance requirements. All of these boxes will need to be ticked to make it feasible to deploy such a solution in a critical infrastructure. To address the requirements, the system will need to provide: 
	
	\begin{itemize}
		\item A deployment solution that is cost-effective, quick and scalable, whilst compatible with existing IT infrastructure; 
		
		\item A number of highly attractive honeypot devices, acting as decoys to lure the attackers away from the real, valuable machines in the deployment network; 
		
		\item A mechanism to isolate the honeypots from other elements of the network, so that in the event of a honeypot being compromised the valuable devices in the network can be detached to protect them; 
		
		\item A centralisation point for data collected from the honeypots, i.e. a monitoring server, giving administrators a holistic view of their system in order to make well-informed decisions; 
		
		\item An effective and reliable alert system, such that the key parties can be alerted when an event occurs in the system.
		
	\end{itemize}

\section{Technologies}
	\subsection{Docker}
	
	Docker is a containerisation technology. As described by Amazon Web Services [https://aws.amazon.com/what-are-containers/ , accessed 24th March 2018], "containers are a method of operating system virtualization that allow you to run an application and its dependencies in resource-isolated processes."
	
		\subsubsection{Benefits of Containerisation}
		
		Docker has revolutionised the way in which many organisations manage their services. By bundling everything that an application needs to run into a single container \textit{image}, the complexity of deploying services is reduced to running the pre-configured image. Containers are the building blocks of microservices, which allow for isolation and decoupling of services into their own units. This approach to deploying services gives:
		
		
		\begin{itemize}
			\item Environmental consistency for applications
			
			\item Rapid deployment and re-deployment of services
			
			\item Scalable deployment through efficiency
			
			\item Version control
		\end{itemize}
		
		\subsubsection{Containers versus Virtual Machines}
		
		- The move towards containers: benefits
		- Examples of migration to containers (NetFlix)
		
		\subsubsection{Networking in Containers}
		

	\subsection{Amazon Web Services}
		
	Using a cloud platform to host enterprise-level systems is highly economical, allowing management of resources and deployment to be outsourced and simplified with the result that they consume less of the focus of the system administrators. The use of a cloud service is a practical solution for hosting the IT infrastructure of large organisations, for a variety of reasons. 
	
	\begin{itemize}
		\item There is a guarantee of maximum uptime, allowing for uninterrupted monitoring;
		
		\item There is no need for upfront capital investment, since there are none of the installation or maintenance costs associated with hardware upgrades or expansion;
		
		\item There is an opportunity for centralisation of all of an organisations systems and data, making it more straight-forward to manage;
		
		\item \textbf{(There are a wide range of geographic locations available for deployment since datacentres	are located all over the world, should this be a requirement)}
		
	\end{itemize}
	With an ever-increasing number of companies hosting their operations in the cloud, a cloud-deployment is a feasible deployment solution for cyber-incident monitoring systems in the long-term.


	\subsection{Elasticsearch, Logstash and Kibana - The ELK Stack}


%%
%% SECTION 1 - The Current Cyber-Threat Landscape
%%
\section{Closely-related Work}

%%A discussion of closely-related projects that have covered similar topics and address similar issues to the work that will be presented in the following chapters.

There is a renewed focus on utilising the unique capabilities of honeypot technologies, in light of the volume of recent and severe cyber-attacks. 

	\subsection{IoTPot}
	
	Studying IoT botnets using targeted honeypot deployments has been investigated by several  researchers. Yin Minn Pa Pa et al. propose IoTPot [6], a reactive, adaptable IoT honeypot that responds to architecture-specific  requests with corresponding architecture-specific responses. 
	
	\subsection{IoTCandyJar}
	
	Another IoT honeypot, IoTCandyJar [7] aims to be an “intelligent-interaction” honeypot, capturing attack data more effectively by utilising machine learning techniques to determine the most appropriate behaviour to prolong attacks. 
	
	However, for both of these solutions the reactivity of the honeypot to probing and attack is the only characteristic studied for effectiveness. 
	
	\subsection{WSN Honeypot}
	
	In contrast, S. Dowling et al. propose ‘Zigbee Honeypot’ [8] to assess IoT cyber- ttacks launched on wireless sensor networks (WSNs). The focus of this implementation is on attracting these attacks, rather than  identifying the characteristics of the Zigbee honeypot that make it most susceptible to attack: There are however a number of useful tactics used to attract the desired attackers. In relation to honeynets, there has been substantially less work done on developing a system suitable for large-scale deployment. 
	
	\subsection{Distributed Virtual Honeynets}
	
	Pisarcík et al. [9] investigate the effectiveness of a distributed virtual honeynet, using OS-level virtualisation. However, their investigation is limited to simulations of botnet DDoS attacks, limiting the applicability of their system to a live production environment. 
	
	Kedrowitsch et al. [10] also look into the benefits and drawbacks of using OS-level virtualisation in honeypot deployments, providing many useful insights on the viability of this approach. 
	
	\subsection{Honeypot-Based Cyber-Incident Monitors}
	
	A honeypot-based cyber-incident monitor is proposed by Vasilomanolakis et al., which recognises the importance of data aggregation and visualisation from multiple honeypots in order to provide important data to key decision makers. [11] This cyber-incident monitor is not targeted at the use- case of critical service infrastructures or low-cost intrusion detection solutions, but addresses the major consideration in this project of centralising important data for key decision-makers. 
	
	Deutsche Telekom (DT) have also developed an open-source honeypot-based visualisation tool called T-Pot [12] using dockerised honeypots to allow organisations to manage honeypot deployments more easily. DT use this framework to manage the security of their own telecommunications infrastructure, and their work on this project strengthens the case for the coupling of scalable honeynet deployments and visualisation tools.  

%\subsection{Project 1}
%Discussion of a closely-related project and a description of its approach with references to the topics discussed above.
