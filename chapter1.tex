\chapter{Introduction}

The section following the chapter title should give an overview of overview of the chapter in 1 to 3 paragraphs. In any document, a title should always be followed by text; a title should never be followed immediately by another title.


\section{Context}

The Context section should let the reader know about the general area in which the dissertation is located.

Paragraphs should start with a general sentence and then explain the sentence. The reader should see what the paragraph is about and then find out the details about the statement of the first sentence. In general, a paragraph should consist of two or more sentences. 

%% Use ~ as non-breakable space between last word and reference
An example of how to reference a figure in the thesis document; see figure~\ref{fig:ImageOfAChick}.

\includefigure{ImageOfAChick}{An Image of a chick}{This caption should describe the figure to the reader and explain to the reader the meaning of the figure. If the interpretation of a figure is left to the reader, the reader will misinterpret the figure!}{image.png}


\section{Overview} 
%% not happy with the name of the section but it will do for now

Giving the reader a short overview of the finding of the dissertation without going into detail, describing briefly the findings of the following chapters and the overall outcome.


\section{Roadmap}

A description of the structure of the dissertation that explains to the reader the contents of the following chapters and the thoughts behind the layout of the dissertation. The chapters following the introduction...