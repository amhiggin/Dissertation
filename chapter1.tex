\chapter{Introduction}

The section following the chapter title should give an overview of overview of the chapter in 1 to 3 paragraphs. In any document, a title should always be followed by text; a title should never be followed immediately by another title.


\section{Problem Area}

%%The Context section should let the reader know about the general area in which the dissertation is located.

%%Paragraphs should start with a general sentence and then explain the sentence. The reader should see what the paragraph is about and then find out the details about the statement of the first sentence. In general, a paragraph should consist of two or more sentences. 

Botnets are continuing to make headlines. They are being used in more diverse ways than ever
before.

Critical service infrastructures are being widely targeted. It is becoming a highly profitable business
for criminals to rent out botnets to political organisations and others (reference booter services for
DDoS - a form of modern-day protest?)
There are more internet connected devices than ever before – more devices to be exploited,
compromised and harnessed for their computational resources. As aptly described by leading
cybersecurity expert Bruce Schneier in his article Botnet of Things [2], “botnets will get larger and
more powerful simply because the number of vulnerable devices will go up by orders of magnitude
over the next few years ... overall, the trends favor the attacker”.

These huge numbers of internet-connected devices are increasingly being operated by individuals
with very limited technical knowledge, and even less awareness of security. These huge numbers of
interconnected devices with poorly implemented security measures are an ideal playground for
hackers.

Honeypots allow the would-be victims of these cybercriminals to learn first-hand about the
techniques and motivations of the attackers. The importance of this kind of active defence
mechanism is growing as attacks are becoming more and more unpredictable.

%% Use ~ as non-breakable space between last word and reference
An example of how to reference a figure in the thesis document; see figure~\ref{fig:ImageOfAChick}.

\includefigure{ImageOfAChick}{An Image of a chick}{This caption should describe the figure to the reader and explain to the reader the meaning of the figure. If the interpretation of a figure is left to the reader, the reader will misinterpret the figure!}{image.png}


\section{Research Objectives} 
%% not happy with the name of the section but it will do for now
%%Giving the reader a short overview of the finding of the dissertation without going into detail, describing briefly the findings of the following chapters and the overall outcome.
This project is being undertaken in a bid to understand the role of honeypot technology in emerging
cybersecurity use-cases – in particular, in tracking and defending against botnets.

The primary objectives of the research are to:

1. Develop a novel, honeypot-based cyber-incident monitoring system to provide effective
intrusion detection and attack mitigation for critical infrastructures; and

2. Obtain a deeper understanding of emerging Internet-of- Things (IoT) botnets using targeted
honeypots deployments, enhancing the knowledge of the wider cybersecurity community.

With the evolution of new threats to the security of data and systems every day as technology
continues to advance, these objectives are conducive to the longer-term fight against cyber-
exploitation in general. Proposing a novel and effective system to address the two objectives is the
central aim of this research.

\section{Contributions}

The contributions of the work to the field of knowledge.

\section{Report Structure and Contents}

A description of the structure of the dissertation that explains to the reader the contents of the following chapters and the thoughts behind the layout of the dissertation. The chapters following the introduction...