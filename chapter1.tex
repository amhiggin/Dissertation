\chapter{Introduction}

The section following the chapter title should give an overview of overview of the chapter in 1 to 3 paragraphs. In any document, a title should always be followed by text; a title should never be followed immediately by another title.


\section{Problem Area}

%%The Context section should let the reader know about the general area in which the dissertation is located.

%%Paragraphs should start with a general sentence and then explain the sentence. The reader should see what the paragraph is about and then find out the details about the statement of the first sentence. In general, a paragraph should consist of two or more sentences. 

Today, there are more systems and devices connected to the internet than ever before as society's reliance on interconnected technologies continually grows. These huge numbers of internet-connected devices are increasingly being operated by individuals with limited technical knowledge, and even less awareness of security. With the scale of systems with poorly-implemented security measures ever-increasing, the modern internet is an ideal playground for hackers - more devices to be compromised, exploited and used for nefarious purposes.

Critical service infrastructures are being widely targeted in attacks, impacting on thousands of people and having major consequences for the operation of organisations. Many of these attacks are being attributed to nation-states across the world: Russia, North Korea and the USA have all variously been accused of being behind attacks that have resulted in entire countries grinding to a halt temporarily. There is no doubt that the increasing role of cyber-attacks in global warfare and terrorism is cause for major concern, and the operators of these infrastructures have new and growing concerns about the safety of their systems. 

There is also a major shift in the hosting of these interconnected services, as organisations continue to migrate to cloud-hosted solutions. The concept of decoupling services by deploying them as individual, containerised microservices has enhanced operational efficiency in organisations. How to translate workable security solutions into these deployments is still largely a work in progress. With the growing scale of migration to cloud-hosted, containerised infrastructure deployments, it is crucial that security catches up.

%% Use ~ as non-breakable space between last word and reference
An example of how to reference a figure in the thesis document; see figure~\ref{fig:ImageOfAChick}.

\includefigure{ImageOfAChick}{An Image of a chick}{This caption should describe the figure to the reader and explain to the reader the meaning of the figure. If the interpretation of a figure is left to the reader, the reader will misinterpret the figure!}{image.png}


\section{Research Objectives} 
%%Giving the reader a short overview of the finding of the dissertation without going into detail, describing briefly the findings of the following chapters and the overall outcome.
The research being undertaken is motivated by the current state of internet-connected systems, changing system infrastructures and global cyberwarfare. In recognition of the gargantuan challenges being faced by the global community in an increasingly connected world, it is clear that there is a need to explore options to both deter and to more effectively defend against exploitation by malicious cyberattackers, whatever their motivations. Exploration of technologies that can be used to learn about the perpetrators of these activities, as well as to defend some of the systems most critical to world peace and human safety, is the driver behind this research. 

The primary objective of the research is to	develop a novel cyber-incident monitoring system to provide effective intrusion detection and attack mitigation for critical infrastructures. It is crucial that such a system be compatible with loosely-coupled service infrastructures that organisations are increasingly adopting, with a view to moving security solutions forward as part of the changing state of systems architectures.

With the evolution of new threats to the security of data and systems every day as technology continues to advance, this objective is conducive to the longer-term fight against cyber-exploitation in general. Proposing a novel and effective system to address the two objectives is the central aim of this research.

\section{Contributions}

Though the project has been limited in scope due to the time constraints associated with an 8-month Masters programme, there have been some substantial contributions to the containerisation of security applications.
\begin{itemize}
	\item This project has seen the development of a dockerised honeynet, a complex system which is bundled into a single deployable unit.
\end{itemize}

\section{Report Structure and Contents}
This report details the state of the art cyber-incident monitoring and cyber-threats in the world today. 

- The formulation of the research problem and subsequent design of the project
- The process of implementation and evaluation of the project
- Concluding with an evaluation and discussion of the findings, along with recommendations for future work in this area.
