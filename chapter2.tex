\chapter{State of the Art}

This chapter should explain the existing work of the areas that your work is based on. The introduction should explain to the reader the area of the work as a whole, how the individual area contribute to the work and what the reader will find in the discussion of each of the areas. The idea is that the reader will be aware of the general contents of the state of the art and will not be surprised by any of the issues that are being discussed.


\section{The Modern Cyber-Threat Landscape}

Explanation of existing work of a given area that describes the area as a whole, how it fits into the work and then breaking it down into components that are relevant to the work.

An example for possible citations~(\cite{Andrew2013empirical}) or by \cite{Asghari2015Economics}.

%%
%% SECTION 1 - The Current Cyber-Threat Landscape
%%
\subsection{Evolution of the Cyber-Threat Landscape}

Starting with a general description of the issue; then drilling down into the details of the issue and how it has been covered in the literature. For a skeleton at the beginning of the writing, this text should be replaced by a general short description, so that you know what you want to discuss and can review the sequence of the discussion.

\subsection{Motivations}

Starting with a general description of the issue; then drilling down into the details of the issue and how it has been covered in the literature. For a skeleton at the beginning of the writing, this text should be replaced by a general short description, so that you know what you want to discuss and can review the sequence of the discussion.

\subsection{The Internet of Things}

Connectivity has been introduced rapidly into processes in critical systems and infrastructures over the past several years. Industries and sectors including agriculture, energy, health and transport have all seen revolutionary transformations thanks to what has been dubbed 'Industry 4.0'. 

Whilst this enhanced communicativity and connectivity between devices and processes has enhanced and optimised industrial processes and enabled the conception of countless new applications, the pace of the transformation has broadly resulted in the neglect of security considerations for these systems. Poor design choices have meant that devices and systems which were never online before are now being exposed to external systems, opening up the potential for them to be significantly tampered with in ways that can affect the welfare of entire populations.  Crucial systems and infrastructures on which society relies on a daily basis such as satellites, power grids, medical devices and traffic control systems are connected to the internet, making them increasingly vulnerable to attacks from anyone who wishes to exploit them.

IoT devices will continue to be one of the most effective and easily exploitable tools for attackers. Until there is a truly commercial, market-driven need to secure IoT devices, manufacturers will continue to avoid the expense of recalling and securing the billions of IoT devices that have been so far exploited by these attackers. This means that unfortunately, the exploitation of IoT devices will
not end any time soon.

%%
%% SECTION 2 - The State of the Art wrt Botnets
%%
\section{Botnets}

Explanation of existing work of a given area that describes the area as a whole, how it fits into the work and then breaking it down into components that are relevant to the work.

An example for possible citations~(\cite{Andrew2013empirical}) or by \cite{Asghari2015Economics}.

%%
%% SECTION 1 - The Current Cyber-Threat Landscape
%%
\subsection{Background}

Starting with a general description of the issue; then drilling down into the details of the issue and how it has been covered in the literature. For a skeleton at the beginning of the writing, this text should be replaced by a general short description, so that you know what you want to discuss and can review the sequence of the discussion.

\subsubsection{Definition}
\subsubsection{Motivations}
Cybercriminals and their supporters leverage botnets for a huge variety of activities. In general they are employed by schemes which require volume to be carried out successfully, whether this is volume of network traffic, computation, or participants.

Some of the most commonly observed uses of botnets are as follows:

\bullet \textbf{Distributed Denial of Service (DDoS)}

DDoS is the most notorious use of botnets, where a number of machines under the control of the botnet are used to simultaneously overwhelm a website or server with traffic, resulting in performance degradation and even failure of the service.

\bullet \textbf{Click fraud}

Botnets can be used to generate artificial user interactions with advertisements on the web, making it appear that real users have clicked on an advertisement for the purpose of generating revenue.

\bullet \textbf{Extortion}

Cyber-extortionists have long used botnets to threaten individuals and organisations in an attempt to obtain payment. This can be anything from threatening a crippling DDoS attack on a company website, to more recent and invasive compromises of home devices. A somewhat questionable endeavour by an anonymous security researcher {http://insecam.com/} is a public website that is hosting footage from thousands of personal, insecure IP cameras in homes and businesses across the world, with the claimed intent being to highlight the potential for extortion by neglecting device security.

\bullet \textbf{Spam}

\bullet \textbf{Password-cracking}

Since botnets allow for synchronised computations across multiple devices, they are often used for brute-force password-cracking computations.

\bullet \textbf{Cryptocurrency Mining}

The explosion of investment in cryptocurrencies over the past number of months has led to a significant increase in the use of botnets to mine cryptocurrencies. One of the most recent trends to emerge has been that of cryptojacking, where visitors of legitimate websites unknowingly execute crypto-mining scripts that have been injected into the webpage - often without the website owner's knowledge through third-party applications. Botnets are also being used for DDoSing of competing groups performing mining.

Make reference to cryptojacking trends!

\subsection{Design}

Starting with a general description of the issue; then drilling down into the details of the issue and how it has been covered in the literature. For a skeleton at the beginning of the writing, this text should be replaced by a general short description, so that you know what you want to discuss and can review the sequence of the discussion

\subsubsection{Architecture}
\subparagraph{Centralised}
\subparagraph{Peer-to-Peer}

\subsection{State of the Art}
Starting with a general description of the issue; then drilling down into the details of the issue and how it has been covered in the literature. For a skeleton at the beginning of the writing, this text should be replaced by a general short description, so that you know what you want to discuss and can review the sequence of the discussion

\subsubsection{Recent Developments}
\subsubsection{Emerging Trends}
\subsubsection{Timeline of Evolution}

\section{Honeypot Technologies}

\subsection{Background}
\subsubsection{Definitions}
Cybersecurity expert and Honeynet Project founder Lance Spitzner defines honeypots as "a security resource who's value lies in being probed, attacked or compromised". They are categorised as an active network defense mechanism, and are deployed solely with the intention that they will be attacked. 

Honeypots can be classified as both decoys and sensors depending on the context of their deployment.

\bullet Decoys

A honeypot will divert an attacker’s attention away from valuable devices in the network,
particularly important in a production environment. In order for this strategy to be effective, the honeypot should be the perfect decoy, appearing to be exactly what the attacker is looking for.

\bullet Sensors

Honeypots always act as sensors in the network in which they are deployed. When attacked, system administrators can be alerted and valuable attack data can also be captured for analysis regarding the attacker’s behaviour, strategies and motivations.

Honeynets TODO define and explain.

\subsubsection{Motivations for Use}
Traditional intrusion-detection mechanisms, such as firewalls, define all attackers passively and simply alert to the fact that a security incident, such as an unauthorised connection attempt, has occurred. Honeypots differ in this regard: They proactively entice attackers to give away their attack strategies and intentions by leaving their tracks behind. By learning what these attackers are targeting, more effective defences can be implemented as vulnerabilities and flaws are identified.

There are a number of specific and significant benefits associated with the use of honeypot technologies.

\bullet Mimic existing systems

\bullet Inside the network

\bullet Extensive Logging Potential

\bullet Goal-oriented

\bullet Few False Positives

A premise of using a honeypot is that nobody should communicate with it: There is no legitimate reason to interact with a honeypot. This means that typically, a honeypot will trigger very few false positives, since every interaction with a honeypot is automatically distrusted. It is important however to realise that in a production environment, individuals within the organisation may mistakenly interact with the honeypot – something which can often be deduced from looking at the logging captured by higher-interaction honeypots. 

\bullet Offensive and Defensive


\subsection{Design}

\subsubsection{Interactivity}

Interactivity levels of honeypots are an important consideration, which define (i) the ability of the attacker to interact with the honeypot, and consequently (ii) the volume and type of information that can be gathered by that honeypot. Levels of interaction range from simply allowing a connection to be made over SSH or another protocol, to being able to download and install malware binaries.  The cost versus learning benefits of honeypot interactivity levels increase proportionally, meaning that a highly interactive honeypot will likely be expensive to host and maintain. 

There are three typical classifications of honeypot interactivity level: Low, high and medium.

\begin{itemize}
	\item Low
	
	These honeypots are most commonly used for intrusion detection, i.e. they are designed to alert someone to the fact that an attack has occurred, but do not provide any means of interacting with the attacker or capturing the attack data. They are used in cases where a lower-risk solution is preferred: In general, a low-interaction honeypot will simply be an emulation of a real service, and so does not offer any opportunity for compromise by an attacker.
	
	\item High
	
	High interaction honeypots are fully-fledged systems, and give the opportunity for attackers to use real applications on these systems. A great deal can be learned about the nature of attacks from using high-interaction honeypots, since it is the closest thing to observing attacks "in the wild". The trade-off here is the high-risk associated with the exposure of a real device to a malicious actor, which in the worst case could see the entire system taken over by an attacker and used to launch further attacks on other devices.
	
	\item Medium
	
	As described, the required level of interactivity of a honeypot with their attacker depends on the use context: Low-interaction devices are commonly used for intrusion detection, whereas high-interaction devices are live systems that can capture detailed information about the behaviour of attackers such as botnets in the wild. Medium interaction honeypots offer a good middle-ground, using emulated systems to allow a level of interactivity with the attacker whilst not exposing any real systems to the attacker. 
	
	
\end{itemize}

\subsubsection{Deployment Scenarios}


\begin{itemize}
	\item Production Honeypots
	
	Production honeypots are generally deployed in large, corporate networks with the intention that they will act as a part of the active network defence in the organisation’s infrastructure. Their primary purpose is to act as a decoy, luring the attacker away from valuable machines on the network with a seemingly more valuable and vulnerable target on the network. This enables the honeypot to alert system administrators to the fact that there has been an intrusion, giving them the chance to isolate the valuable devices in the network from the infected honeypot. It also enables system administrators to identify vulnerable points in their infrastructure, allowing them to continuously improve the security of their systems. 
	
	The idea of attracting attackers into the system, making them interact with it and give away their attack strategies without causing any harm to the real systems is one of the major attractions of using honeypots in a production environment. The solution being proposed as part of this research is targeted at use in production environments for exactly this reason, thwarting the attacker’s intended compromise of the network.
	
	
	\item Research Honeypots
	
	As the name suggests, these honeypots are generally deployed for the purpose of research rather than as a security measure to protect a network. The emphasis with these honeypots is not so much on the ability of the honeypot to act as a decoy node in a network, and more on the ability of the honeypot to collect valuable data for analysis. By allowing attackers to interact with and infect the honeypot, research can be carried out relating to the behaviour and strategies of the attackers.
	
\end{itemize}

\subsubsection{Applications}

\subsection{State of the Art}

\subsubsection{Honeypots of Note}

At the time of writing, there were over 1,000 open-source honeypot projects listed on GitHub.

\begin{enumerate}
	
	\item Kippo
	
	Kippo is a honeypot written in Python which is no longer under active development, but whose direct descendants include the Cowrie honeypot. It is a medium-interaction SSH honeypot, which is designed to capture the entire session interaction of a connected attacker. It is an emulation of a Debian Linux installation, which provides a fake file system and shell for each attacker to interact with while logging all attack data – including records of any downloaded malware. 
	
	Kippo includes configuration options for permitted usernames and password combinations, and all brute-force attempts are logged.
	As quoted from desaster, creater of Kippo, "By running kippo, you're virtually mooning the attackers. Just like in real life, doing something like that, you better know really well how to defend yourself!"
	
	
	\item Cowrie
	
	Cowrie is a medium interaction honeypot which was originally forked from the Kippo honeypot project. It is under active development and is widely used as an SSH/Telnet honeypot solution, providing all of the features that the Kippo honeypot provided and more. 
	
	It has many useful capabilities and features, including the following:
	
	\begin{itemize}
		\item It is actively maintained by a dedicated community, with quick response times to queries and bug-fixes.
		
		\item It enables attackers to gain access to the honeypot using SSH and Telnet, both protocols widely exploited by IoT botnets.
		
		\item Cowrie fully emulates a Debian installsation, with an out-of-the-box configurable filesystem that an attacker can interact with.
		
		\item It is an open-source project, allowing users to adapt the honeypot to suit their specific needs.
		
		\item Cowrie records interactions between the attacker and the honeypot, logging everything from commands executed and file downloads attempted to the source IP addresses and protocol information.
		
		\item Cowrie does not limit the number of simultaneous sessions, and can present a mock filesystem and shell to each attacker independently.
		
		\item The fact that the Cowrie software doesn't call on any external sofware to operate makes it much less vulnerable to third-party compromises. It also improves substantially on its predecessor, Kippo, in that many of the fingerprinting issues are resolved.
		
		
		Cowrie cannot execute malware, but can be used in conjunction with other solutions such as Cuckoo Sandbox in order to execute malware safely in a controlled environment.
		
	\end{itemize}
	
	\item Dionaea
	
	Dionaea is a commonly used low-interaction honeypot. It is a medium-interaction honeypot, which emulates a vulnerable Windows operating system running protocols like SSH, HTTP and FTP. The stated goal of Dionaea is to trap malware and obtain a copy of it, to allow it to be inspected using other software such as Cuckoo Sandbox.
	
	
	\item Honeyd 
	
\end{enumerate}

\subsubsection{Challenges}
There are many challenges to overcome in order to make the use of honeypots more widespread. To many, the idea of inviting a malicious actor into a production system seems like a potentially destructive action, and so many system administrators do not consider honeypots as part of their security infrastructure. Canary, a honeypot solution developed by Thinkst Applied Research,  highlight on their product landing page the challenge behind adopting honeypots in large networks: “With all the network problems we have, nobody needs one more machine to administer and worry about”. This statement highlights the demand and lack of supply of feasible solutions for large organisations regarding deployment and maintenance of honeypot-based systems. 
 

\section{Summary}

This section should summarize the essential points of the state-of-the-art and give the reader an overview of relevant projects; ideally, this can be summed up by providing a table (see table~\ref{table:tablelabel}) with the relevant projects and issues that they address.

\begin{table}[!h]
\begin{center}
	\begin{tabular}{|c|c|c|c|} 
	\hline
 	\bf Column 1  & \bf Column 2  & \bf Column 3 & \bf Column 4  \\
  	\hline
	Row 1 & Item 1 & Item 2 & Item 3 \\
	Row 2 & Item 1 & Item 2 & Item 3 \\
	Row 3 & Item 1 & Item 2 & Item 3 \\
	Row 4 & Item 1 & Item 2 & Item 3 \\
	\hline
	\end{tabular}
\end{center}
\caption[ToC Caption]{Caption that explains the table to the reader}	
\label{table:tablelabel}
\end{table}