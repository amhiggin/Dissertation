\chapter{State of the art}

This chapter should explain the existing work of the areas that your work is based on. The introduction should explain to the reader the area of the work as a whole, how the individual area contribute to the work and what the reader will find in the discussion of each of the areas. The idea is that the reader will be aware of the general contents of the state of the art and will not be surprised by any of the issues that are being discussed.


\section{Area 1}

Explanation of existing work of a given area that describes the area as a whole, how it fits into the work and then breaking it down into components that are relevant to the work.

An example for possible citations~(\cite{Andrew2013empirical}) or by \cite{Asghari2015Economics}.


\subsection{Issue A}

Starting with a general description of the issue; then drilling down into the details of the issue and how it has been covered in the literature. For a skeleton at the beginning of the writing, this text should be replaced by a general short description, so that you know what you want to discuss and can review the sequence of the discussion.

\subsection{Issue B}

Starting with a general description of the issue; then drilling down into the details of the issue and how it has been covered in the literature. For a skeleton at the beginning of the writing, this text should be replaced by a general short description, so that you know what you want to discuss and can review the sequence of the discussion.

\subsection{Issue C}

Starting with a general description of the issue; then drilling down into the details of the issue and how it has been covered in the literature. For a skeleton at the beginning of the writing, this text should be replaced by a general short description, so that you know what you want to discuss and can review the sequence of the discussion.


\section{Closely-related Projects}

A discussion of closely-related projects that have covered similar topics and address similar issues to the work that will be presented in the following chapters.

\subsection{Project 1}

Discussion of a closely-related project and a description of its approach with references to the topics discussed above.
 

\section{Summary}

This section should summarize the essential points of the state-of-the-art and give the reader an overview of relevant projects; ideally, this can be summed up by providing a table (see table~\ref{table:tablelabel}) with the relevant projects and issues that they address.

\begin{table}[!h]
\begin{center}
	\begin{tabular}{|c|c|c|c|} 
	\hline
 	\bf Column 1  & \bf Column 2  & \bf Column 3 & \bf Column 4  \\
  	\hline
	Row 1 & Item 1 & Item 2 & Item 3 \\
	Row 2 & Item 1 & Item 2 & Item 3 \\
	Row 3 & Item 1 & Item 2 & Item 3 \\
	Row 4 & Item 1 & Item 2 & Item 3 \\
	\hline
	\end{tabular}
\end{center}
\caption[ToC Caption]{Caption that explains the table to the reader}	
\label{table:tablelabel}
\end{table}
