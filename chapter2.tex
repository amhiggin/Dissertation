\chapter{State of the Art}

%%This chapter should explain the existing work of the areas that your work is based on. The introduction should explain to the reader the area of the work as a whole, how the individual area contribute to the work and what the reader will find in the discussion of each of the areas. The idea is that the reader will be aware of the general contents of the state of the art and will not be surprised by any of the issues that are being discussed.
This section is intended as an introduction to the state of the art in cybersecurity and cyber-threats in 2018, in order to give the reader context on the area in which the research fits.

Section 1 outlines the cyber-threat landscape across the world as it appears in 2018. Motivations for and challenges presented by cybercrime and cyberwarfare in the current day are investigated, with a view to highlighting the immense importance of cybersecurity in society today.

Section 2 delves into the detail of one of the biggest emerging threats to security today: Botnets. The state of the art in their design and motivations is explored, with a focus on the major evolving threat of Internet-of-Things botnets.

Finally, section 3 considers the role of effective intrusion detection in securing systems and devices. The role of honeypot technology in particular is discussed in the context of existing solutions, with a detailed analysis of the importance characteristics of their design in order to employ them effectively.

%% Botnets constitute one of the most serious weapons at the disposal of criminals today, and their rapid evolution in recent years is highlighted. Finally, the role of honeypots in effective intrusion detection and active network defence is investigated


\section{The Modern Cyber-Threat Landscape}

%%Explanation of existing work of a given area that describes the area as a whole, how it fits into the work and then breaking it down into components that are relevant to the work.
%%An example for possible citations~(\cite{Andrew2013empirical}) or by \cite{Asghari2015Economics}.

Most significant cyber incidents that have occurred in recent times did so on a global scale, extending across all physical borders and legal jurisdictions and affecting a diverse set of victims. Across the world, cyber-attacks are playing a more prominent role in warfare and terrorism than ever before.

brings to the fore the importance of developing effective defence and counter-attack strategies. [1]  A notable trend is that the motivations for cyber-attacks are becoming increasingly political, with high-profile victims being widely targeted. The United States are currently locked in a cyber-combat against “HIDDEN COBRA”, a codename for what they claim is North Korea’s DDoS botnet infrastructure, as part of ongoing disputes between the two nations. A US national alert [3] issued in  3 June 2017 by the US-CERT1 claimed that HIDDEN COBRA had “likely targeted the aerospace, telecommunications, and finance industries” in the United States. Earlier this year in February 2017, an ISIS-linked group of hackers infiltrated six websites affiliated with the UK National Health Service (NHS), posting graphic images of violence in Syria “in retaliation for the West’s aggression in the Middle East” [4] – further confirmation of the role of cyber-attacks in modern terrorism. 


	%%
	%% SECTION 1 - The Current Cyber-Threat Landscape
	%%
	\subsection{Evolution of the Cyber-Threat Landscape}
	
	Starting with a general description of the issue; then drilling down into the details of the issue and how it has been covered in the literature. For a skeleton at the beginning of the writing, this text should be replaced by a general short description, so that you know what you want to discuss and can review the sequence of the discussion.
	
	\subsection{Motivations}
	
	•	Cyber-Crime
	
	o	Cyberterrorism
	•	Others
	o	“Script-Kiddies”
		Definition
		Reusing other people’s malware
		An example of a recent script-kiddie attack which had a big impact



		\subsubsection{Politics and Terrorism}
		•	Politically Motivated Cyber-attacks
		o	DDoS attacks on political campaigns
		
		\subsubsection{Financial Gain}
		o	Financially motivated crime. Extortion, social engineering etc.
		
		\subsubsection{Competition}
		
		\subsubsection{Insiders}
		o	Insiders
			Intentional: Leaks, insider attacks
			Unintentional: Social engineering (phishing etc).

	\subsection{The Internet of Things}
	
	Connectivity has been introduced rapidly into processes in critical systems and infrastructures over the past several years. Industries and sectors including agriculture, energy, health and transport have all seen revolutionary transformations as part of what has been dubbed 'Industry 4.0'. 
	
	Whilst this enhanced communicativity and connectivity between devices and processes has enhanced and optimised industrial processes and enabled the conception of countless new applications, the pace of the transformation has broadly resulted in the neglect of security considerations for these systems. Poor design choices have meant that devices and systems which were never online before are now being exposed to external systems, opening up the potential for them to be significantly tampered with in ways that can affect the welfare of entire populations.  Crucial systems and infrastructures on which society relies on a daily basis such as satellites, power grids, medical devices and traffic control systems are connected to the internet, making them increasingly vulnerable to attacks from anyone who wishes to exploit them.
	
	IoT devices will continue to be one of the most effective and easily exploitable tools for attackers. Until there is a truly commercial, market-driven need to secure IoT devices, manufacturers will continue to avoid the expense of recalling and securing the billions of IoT devices that have been so far exploited by these attackers. This means that unfortunately, the exploitation of IoT devices will
	not end any time soon.
	
	\subsection{Attacks on Critical Infrastructures}
	
	In a report published in 2017 (\textit{J. Brenner, “Keeping America Safe: Towards more Secure Networks for Critical Sectors,” M.I.T. Internet Policy Research Initiative, Cambridge, Massachussets, 2017.}), M.I.T. researcher Joel Brenner states that one of the eight major challenges facing the United States government currently is to “enable critical infrastructure operators to quickly identify and respond to cyber risk arising from cross-sector linkages as well as from their own network”. This concern is well-founded, and is something that is being reflected in incidents across the world with an increasing number of attacks on high-profile targets.
	
	It is clearer than ever that key organisations and critical services require a renewed focus on defence strategies for their own IT infrastructures. The materialisation of the WannaCry ransomware attacks in May 2017, which successfully compromised the systems of high-profile targets in over 150 countries, knocked systems in the UK National Health Service (NHS) at 37 sites offline for over a week with more than 6,912 appointments cancelled in that time. (\textit{Sir Amyas Morse KCB Comptroller and Auditor General National Audit Office UK, “Investigation: WannaCry Cyber Attack and the NHS,” National Audit Office UK, London, 2017.
}) Being such a high-impact incident affecting one of the most critical , this event has emphasised the severe consequences of such an attack on a critical service.

	


%%
%% SECTION 1 - BOTNETS
%%
\section{Botnets}

Botnets are one of many forms of attacker on the web. They propagate as scripted attacks which seek to infect as many devices as possible, harnessing their computational resources to perform a specific function as a collective group. Cyberattacks know no geographical boundaries. Botnets in particular are highly geographically dispersed, requiring coordination to for large attacks. (Note: this is one of the issues that contributes to applying patches to vulnerable IoT devices).

Botnets are not a new research area, but their rapid and continuous evolution means that research is always trying to catch up with the latest techniques and strategies being employed by their so-called “botmasters”.


%%
%% SECTION 1 - The Current Cyber-Threat Landscape
%%
\subsection{Background}

Starting with a general description of the issue; then drilling down into the details of the issue and how it has been covered in the literature. For a skeleton at the beginning of the writing, this text should be replaced by a general short description, so that you know what you want to discuss and can review the sequence of the discussion.

\subsubsection{Definition}
\subsubsection{Motivations}
Cybercriminals and their supporters leverage botnets for a huge variety of activities. In general they are employed by schemes which require volume to be carried out successfully, whether this is volume of network traffic, computation, or participants.

Some of the most commonly observed uses of botnets are as follows:

\bullet \textbf{Distributed Denial of Service (DDoS)}

DDoS is the most notorious use of botnets, where a number of machines under the control of the botnet are used to simultaneously overwhelm a website or server with traffic, resulting in performance degradation and even failure of the service.

\bullet \textbf{Click fraud}

Botnets can be used to generate artificial user interactions with advertisements on the web, making it appear that real users have clicked on an advertisement for the purpose of generating revenue.

\bullet \textbf{Extortion}

Cyber-extortionists have long used botnets to threaten individuals and organisations in an attempt to obtain payment. This can be anything from threatening a crippling DDoS attack on a company website, to more recent and invasive compromises of home devices. A somewhat questionable endeavour by an anonymous security researcher {http://insecam.com/} is a public website that is hosting footage from thousands of personal, insecure IP cameras in homes and businesses across the world, with the claimed intent being to highlight the potential for extortion by neglecting device security.

\bullet \textbf{Spam}

\bullet \textbf{Password-cracking}

Since botnets allow for synchronised computations across multiple devices, they are often used for brute-force password-cracking computations.

\bullet \textbf{Cryptocurrency Mining}

The explosion of investment in cryptocurrencies over the past number of months has led to a significant increase in the use of botnets to mine cryptocurrencies. One of the most recent trends to emerge has been that of cryptojacking, where visitors of legitimate websites unknowingly execute crypto-mining scripts that have been injected into the webpage - often without the website owner's knowledge through third-party applications. Botnets are also being used for DDoSing of competing groups performing mining.

Make reference to cryptojacking trends!

\subsection{Design}

Starting with a general description of the issue; then drilling down into the details of the issue and how it has been covered in the literature. For a skeleton at the beginning of the writing, this text should be replaced by a general short description, so that you know what you want to discuss and can review the sequence of the discussion

\subsubsection{Architecture}
\subparagraph{Centralised}
\subparagraph{Peer-to-Peer}

\subsubsection{Attack Types}


\subsection{Recent Trends}

%% TODO talk about Internet of Things botnets in particular.
%% Talk about the fact that security on these devices has been very poorly thought through: using Telnet and SSH with insecure passwords is one of the most rudimentary exploits imaginable!

Some of the best-defended targets were threatened by huge numbers of some of the least powerful hosts – insecure IoT devices whose credentials could be easily abused.

\begin{itemize}
	\item \textbf{Mirai}
	
	Mirai is a worm-like family of botnet malware that targets IoT devices. It scans IP address ranges and uses brute-force authentication over the SSH and Telnet protocols to gain access to the device, using a hard-coded dictionary of credentials. Once inside, the malware listens on the device for commands from a C&C server, and continuously scans the network to propagate the infection to otherdevices. Interestingly, other malware running on the device is also killed off by the Mirai malware. 
	
	The botnet first surfaced in 2016, when it {ABCXYZ... describe what happened}. The victims targeted by the Mirai botnet “ranged from game servers, telecoms, and anti-DDoS providers, to political websites and relatively obscure Russian sites”.
		
	Just a few months ago, the authors were charged with “creating and operating two botnets, which targeted “Internet of Things” (IoT) devices”.
\end{itemize}

\subsection{Timeline of Evolution}
%%%% This can be the taxonomy section






%%
%%	Section 2: HONEYPOTS
%%
%%


\section{Honeypot Technologies}

Cybersecurity expert and Honeynet Project founder Lance Spitzner defines honeypots as "a security resource who's value lies in being probed, attacked or compromised". They are categorised as an active network defense mechanism, and are systems which deployed solely with the intention that they will be attacked. 

A key characteristic of honeypots is that they should be attractive to an attacker. 'Attractive' in this context means that the honeypot should appear to be exactly the device that the attacker is looking for: A device that has the potential to be easily exploited, whilst offering maximum value to the attacker.

A secondary definition of honeypots would define them based on the context of their deployment as being either a decoy or a sensor.

\bullet \textbf{Honeypots as Decoys}

A honeypot can be used to divert an attacker’s attention away from the valuable devices in the network in which it is deployed. This is particularly useful in a production environment, when there is a need to protect important devices on the network. 

In order for this strategy to be effective, the honeypot should appear to be exactly what the attacker is looking for. By then being attacked, the honeypot can immediately notify system administrators of an intruder being present in the network, allowing them to take the appropriate course of action to secure the network.

\bullet \textbf{Honeypots as Sensors}

Honeypots can also be viewed as sensors in the network in which they are deployed, since they can collect valuable data about the attacks that they receive. This is particularly useful for detecting weaknesses in system design, since the captured attack data can be analysed to understand the attacker’s behaviour, strategies and motivations.

\subsection{Design}

\subsubsection{Interactivity}

Interactivity levels of honeypots are an important consideration, which define (i) the ability of the attacker to interact with the honeypot, and consequently (ii) the volume and type of information that can be gathered by that honeypot. Levels of interaction range from simply allowing a connection to be made over SSH or another protocol, to being able to download and install malware binaries.  The cost versus learning benefits of honeypot interactivity levels increase proportionally, meaning that a highly interactive honeypot will likely be expensive to host and maintain. 

Any effective honeypot must be capable of interacting at some level with an attacker, while also quietly monitoring their actions. There are three practical classifications of honeypot interactivity level: Low, high and medium.

\begin{itemize}
	\item Low
	
	These honeypots are most commonly used for intrusion detection, i.e. they are designed to alert someone to the fact that an attack has occurred, but do not provide any means of interacting with the attacker or capturing the attack data. They are used in cases where a lower-risk solution is preferred: In general, a low-interaction honeypot will simply be an emulation of a real service, and so does not offer any opportunity for system compromise by an attacker.
	
	\item High
	
	High interaction honeypots are at the other end of the spectrum compared to low interaction honeypots. They are fully-fledged systems, and give the opportunity for attackers to use real applications during their interactions. A great deal can be learned about the nature of attacks from using high-interaction honeypots, since it is the closest thing to observing attacks "in the wild". The trade-off is the high-risk associated with the exposure of a real device to a malicious actor, which in the worst case could see the entire system taken over by an attacker and used to launch further attacks on other devices.
	
	\item Medium
	
	As described, the required level of interactivity of a honeypot with their attacker depends on the use context: Low-interaction devices are commonly used for intrusion detection, whereas high-interaction devices are live systems that can capture detailed information about the behaviour of attackers such as botnets in the wild. Medium interaction honeypots offer a good middle-ground, using emulated components of a real system to allow a level of interactivity with the attacker whilst not exposing any real systems that could be compromised.
	
	
\end{itemize}



\subsubsection{Deployment Scenarios}
There are primarily two deployment scenarios for honeypots: Production and research.

\begin{itemize}
	\item \textbf{Production Honeypots}
	
	Production honeypots are generally deployed in large, corporate networks with the intention that they will act as a part of the active network defence in the organisation’s infrastructure. Their primary purpose is to act as a decoy, luring the attacker away from valuable machines on the network with a seemingly more valuable and vulnerable target on the network. This enables the honeypot to alert system administrators to the fact that there has been an intrusion, giving them the chance to isolate the valuable devices in the network from the infected honeypot. It also enables system administrators to identify vulnerable points in their infrastructure, allowing them to continuously improve the security of their systems. 
	
	The idea of attracting attackers into the system, making them interact with it and give away their attack strategies without causing any harm to the real systems is one of the major attractions of using honeypots in a production environment. The solution being proposed as part of this research is targeted at use in production environments for exactly this reason, thwarting the attacker’s intended compromise of the network.
	
	
	\item \textbf{Research Honeypots}
	
	As the name suggests, these honeypots are generally deployed for the purpose of research rather than as a security measure to protect a network. The emphasis with these honeypots is not so much on the ability of the honeypot to act as a decoy node in a network, and more on the ability of the honeypot to collect valuable data for analysis. By allowing attackers to interact with and infect the honeypot, research can be carried out relating to the behaviour and strategies of the attackers.
	
\end{itemize}


\subsubsection{Motivations for Use}

When compared to traditional intrusion-detection mechanisms such as firewalls, there are some crucial difference. Firewalls define all attackers passively and simply alert to the fact that a security incident, such as an unauthorised connection attempt, has occurred. 

Honeypots differ in this regard: They proactively entice attackers to give away their attack strategies and intentions by leaving their tracks behind on the device. By learning what these attackers are targeting, more effective defenses can be implemented as vulnerabilities and flaws are identified.

There are a wide range of significant benefits associated with the use of honeypot technologies. 

\bullet \textbf{Configurability}

In general, honeypots are highly configurable and can be made to mimic real systems. This configurability allows those deploying them to continuously improve their defense strategies by adapting to evolving behaviours, as well as providing a means of observing attackers in the wild without being detected.

\bullet \textbf{Inside the Network} 

Honeypots are deployed inside the systems infrastructure of the system they are protecting, rather than on the fringes as with firewalls. This allows for security much closer to the real systems that are being targeted. 

\bullet \textbf{Logging Potential} 

As described, there is huge potential to capture valuable information about the nature of attacks being launched against a system, allowing system administrators to stay up-to-date with the security requirements of their systems.

\bullet \textbf{Few False Positives}

The premise of using a honeypot is that nobody should communicate with it: There is no legitimate reason to interact with a honeypot, since it is deployed with the sole purpose of attracting attacks. This means that typically, a honeypot will trigger very few false positives, since every interaction with a honeypot is automatically distrusted. It is important however to realise that in a production environment, individuals within the organisation may mistakenly interact with the honeypot – something which can often be deduced from looking at the logging captured by higher-interaction honeypots. 

\bullet \textbf{Offensive and Defensive}



\subsubsection{Honeynets}
%% Describe honeynets and how they can be beneficial as part of the network security of a system.
Honeynets are networks of interconnected honeypots, which coordinate their efforts to provide active network defense.

\subsection{State of the Art}

\subsubsection{Honeypots of Note}

At the time of writing, there were over 1,000 open-source honeypot projects listed on GitHub. {https://github.com/search?utf8=%E2%9C%93&q=honeypot&type= , 24th March 2018 }

\begin{enumerate}
	
	\item \textbf{Kippo}
	
	Kippo is a honeypot written in Python which is no longer under active development, but which has influenced many subsequent honeypot projects. It is a medium-interaction SSH honeypot, which is designed to capture the entire session interaction of a connected attacker. It is an emulation of a Debian Linux installation, which provides a fake file system and shell for each attacker to interact with while logging all attack data – including records of any downloaded malware. 
	
	Kippo offers a falsified SSH service to which attackers can connect, and includes configuration options for permitted usernames and password combinations. All brute-force login attempts and interactions with the honeypot are logged. 
	
	(\textit{Refer to the fingerprinting of Kippo which resulted in its eventual retirement})
	
	\item \textbf{Cowrie}
	
	Cowrie is a medium interaction honeypot which was originally forked from the Kippo honeypot project. It is under active development and is widely used as an SSH/Telnet honeypot solution, providing all of the features that the Kippo honeypot provided and more. 
	
	It has many useful capabilities and features, including the following:
	
	\begin{itemize}
		\item It is actively maintained by a dedicated community, with quick response times to queries and bug-fixes.
		
		\item It enables attackers to gain access to the honeypot using SSH and Telnet, two protocols which are widely targeted by IoT botnets.
		
		\item Cowrie fully emulates a Debian installsation, with an out-of-the-box configurable filesystem that an attacker can interact with.
		
		\item It is an open-source project, allowing users to adapt the honeypot to suit their specific needs.
		
		\item Cowrie records interactions between the attacker and the honeypot, logging everything from commands executed and file downloads attempted to the source IP addresses and protocol information.
		
		\item Cowrie does not limit the number of simultaneous sessions, and can present a mock filesystem and shell to each attacker independently.
		
		\item The fact that the Cowrie software doesn't call on any external sofware to operate makes it much less vulnerable to third-party compromises. It also improves substantially on its predecessor, Kippo, in that many of the fingerprinting issues are resolved.
	\end{itemize}

	Cowrie cannot execute malware, but can be used in conjunction with other solutions such as Cuckoo Sandbox {https://github.com/cuckoosandbox/cuckoo , accessed 24th March 2018} in order to execute and analyse malware safely in a controlled environment.
	
	\item Dionaea
	
	Dionaea is a commonly used low-interaction honeypot. It is a medium-interaction honeypot, which emulates a vulnerable Windows operating system running protocols like SSH, HTTP and FTP. The stated goal of Dionaea is to trap malware and obtain a copy of it, to allow it to be inspected using other software such as Cuckoo Sandbox.
	
	
	\item Honeyd 
	
\end{enumerate}

\subsubsection{Challenges}
	\subparagraph{Detection}
	One of the major considerations when designing or using a honeypot-driven solution is the ability of the honeypot to be detected as a non-genuine system. This detection is known as \textit{fingerprinting}, and it is a continuous challenge for honeypot designers to combat fingerprinting checks by many different attackers.
	
	(\textit{Example of fingerprinting by Hajime?})

	\subparagraph{Adoption}
	There are many challenges to overcome in order to make the use of honeypots more widespread. To many, the idea of inviting a malicious actor into a production system seems like a potentially destructive action, and so many system administrators do not consider honeypots as part of their security infrastructure. Canary, a honeypot solution developed by Thinkst Applied Research,  highlight on their product landing page the challenge behind adopting honeypots in large networks: “With all the network problems we have, nobody needs one more machine to administer and worry about”. This statement highlights the demand and lack of supply of feasible solutions for large organisations regarding deployment and maintenance of honeypot-based systems. 
	
	\subparagraph{Ethical Concerns}
 	The use of any surveillance technology, which is the category into which honeypots fall, has ethical questions associated with it. Honeypots are widely accepted as an ethical approach to intrusion detection and understanding cyber-attacks. Since there is no legitimate reason for interacting with a honeypot, any interactions are likely to be of a malicious nature – in	which case, the honeypot simply serves to counteract further attacks of this type in the	future. As outlined by the SANS Technology Institute [1], their use does not involve:
 	
 	\begin{itemize}

 		\item Entrapment, since there is no inducement to attack the honeypot – rather, it is	something that all devices are vulnerable to because attackers will always seek to attack devices within their reach, regardless of whether it may be a honeypot. As aptly phrased by security researcher Lance Spitzner [2], “Attackers find and break into honeypots on their own initiative”.
 	
 		\item Invasion of privacy, since surveillance mechanisms are broadly seen to be acceptable where they serve simply to protect their own environment.
 		
	\end{itemize}

	With reference to the research being undertaken in this project, ethical considerations are discussed later in the Design chapter (Chapter 4).


\section{Summary}

This section should summarize the essential points of the state-of-the-art and give the reader an overview of relevant projects; ideally, this can be summed up by providing a table (see table~\ref{table:tablelabel}) with the relevant projects and issues that they address.

\begin{table}[!h]
\begin{center}
	\begin{tabular}{|c|c|c|c|} 
	\hline
 	\bf Column 1  & \bf Column 2  & \bf Column 3 & \bf Column 4  \\
  	\hline
	Row 1 & Item 1 & Item 2 & Item 3 \\
	Row 2 & Item 1 & Item 2 & Item 3 \\
	Row 3 & Item 1 & Item 2 & Item 3 \\
	Row 4 & Item 1 & Item 2 & Item 3 \\
	\hline
	\end{tabular}
\end{center}
\caption[ToC Caption]{Caption that explains the table to the reader}	
\label{table:tablelabel}
\end{table}