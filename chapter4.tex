\chapter{Design}

Description of the Design chapter and its contents.


\section{Proposed Work}

There has also been a lack of conclusive research carried out to identify the characteristics that make a honeypot the target of an IoT botnet, particularly since this is such a recently realised threat. The looming threat of IoT botnet attacks on infrastructures of critical services makes it all the more important to understand these attackers and how they identify their weakest targets. Again, the configurability and informative capabilities of honeypots makes them an attractive option here. These conclusions have motivated the formulation of the proposed work, outlined in the following sections. A discussion of the relevance of and lessons learned from some Closely Related Work is also included towards the end of the section.
2. The Study of IoT Botnets using Honeypot

\subsection{Cyber-Incident Monitor for Critical Infrastructures}
%% TODO add in Jason's suggestion of how the jumpbox device in GP's office propagates the infection into the core of the system

The intention is to deploy a network of honeypots (known as a ‘honeynet’) deployed in “containers7
”
on cloud-hosted virtual machine instances, mimicking the setup of a home IoT device network. This
is a network designed to be attacked: IoT home networks are likely to attract an IoT botnet, since
compromising devices in these environments generally require unsophisticated attacks. By hosting
the system on cloud instances, infections can be isolated from physical machines and maintenance
becomes much simpler.
The honeypots in this system will act as:
● Sensors, providing a feedback mechanism for the network in relation to attack activities by
generating alerts and logging attack data for analysis;
● Decoys, diverting the attacker’s attention away from the valuable devices while at the same
time capturing valuable information about the attack strategies and behaviours of IoT
botnets.
The manner in which the proposed solution addresses the research objectives is outlined in the
following sections.

Information gathered from the honeynet will be communicated to a centralised monitoring system,
where the data can be aggregated, analysed and visualised. This is a crucial component of the cyberincident
monitoring system, where the attack data is gathered centrally and made accessible to the
management user. The monitoring device should be directly contactable from the honeypots, but
obscured from attackers. If necessary, it should be able to disconnect infected honeypots from the
network, preventing the propagation of infection to other devices. This will require some further
design decisions to be made during implementation in relation to the network architecture and
communication protocols used: Candidates include SNMP, Yang and TLS.
A cloud infrastructure will be used to host the various services (i.e. honeypots, monitor system) on
virtual machines. Each of these machines should host multiple OS-level containers, internetworked
to appear as though they are individual machines within the same network.
By deploying the honeypots in internetworked containers it becomes possible to provide a flexible,
low-cost, low-maintenance deployment solution for organisations; each container will only use a
relatively small proportion of a machine’s CPU power. Importantly, they be deployed easily

7 According to Docker [32], “a container image is a lightweight, stand-alone, executable package of a piece of
software that includes everything needed to run it: code, runtime, system tools, system libraries, settings.”
8
alongside existing infrastructure. Increasingly, there is a move towards the use of containers in
honeypot solutions since common fingerprinting mechanisms used by attackers for identifying
virtual honeypots can be avoided, [22] lending credibility to the application of this technology in the
context of the proposed research.


\subsection{Tracking Internet-of-Things Botnets}

From the outset, the most interesting and relevant category of botnet was IoT botnets. Finding a honeypot solution that would

\section{Design Decisions}
	\subsection{Considerations}
		\subsubsection{Choice of Hosting Solution}
	
		AWS EC2 is very flexible and offered some free credits through their AWS Educate programme. Though other solutions like DigitalOcean and Trinity College's OpenNebula infrastructures were considered, the flexibility of options offered by AWS resulted in it ultimately being the platform of choice.
		
		\subsubsection{Choice of Honeypot}
				
		\subsubsection{Network Design}
		
		\subsubsection{Intrusion Alerting Mechanism}
		PSAD
		- About PSAD
		- Enables the sending of emails from the honeypot instance based on threshold rules configured
		\subsubsection{Remote Cyber-Monitoring}
		- PSAD
			- Provided by the local machine (not the remote management instance)
		- Filebeat
		- Management Instance: independent, isolated machine dedicated to remote monitoring and providing visualisation information.

		\subsubsection{Container Capabilities}
		Importance of considering how restricted the container capabilities should be in order to capture attacks, whilst also keeping the systems secure.
		
		As quoted from desaster, creater of Kippo, "By running kippo, you're virtually mooning the attackers. Just like in real life, doing something like that, you better know really well how to defend yourself!"
		
		If an attacker was to discover that they were in fact being monitored by a honeypot, it is plausible that they could launch an attack on the honeypot or the platform on which it is hosted.
		

		\subsubsection{Deployment Replicability}
		
		A consideration in any project is the reproducability of an experimental setup, in order that another individual can prove or disprove the findings of research in the future. Computer science projects are notorious for being difficult to verify, and so planning for a reproducable research environment was a key consideration in the design of the project.
		
		One of the significant benefits of container technologies is the ease with which identically configured environments can be deployed. This, coupled with the automation of environment setup by defining a set of bash scripts, ensures the maximum reproducability of the research environment used. 
		
		\subsubsection{Ethical Considerations}
		There are a number of potential ethical issues associated with a project using honeypots, as outlined in Section \textit{(Refer to section)}. A further ethical concern is the ability for a honeypot to be used as an attack platform. After deliberation on the potential for this to arise during the research, the following conclusions were drawn: 	
		
		\begin{itemize}
		\item This risk can be safely viewed as the general risk associated with using devices connected to the web, and not necessarily related to the use of honeypots. This view is in line with that held by Nawrocki et al. (M. Nawrocki, M. Wahlisch, T. C. Schmidt, C. Keil, J. Schonfelder, “A Survey on Honeypot Software and Data Analysis,” in ArXiv e-prints, eprint 1608.06249, 2016.).
		
		\item Although it is never possible to entirely eradicate the risk of attack propagation, the design of the systems being used provide mitigation against this risk.
			\begin{itemize}

			\item The primary honeypot being used is the Cowrie [4] honeypot, an emulated	environment. It does not give any ability for a malicious user make outbound network connections, and so cannot be used to propagate an attack.
			\item \textbf{A secondary honeypot type being used is a ‘jumpbox’ router honeypot which	has the ability to make outbound network connections. However, it is being designed to be a maximally restricted environment so that the probability of attack propagation is also minimal.}
		
			\item Dedicated research environments where the honeypots are sandboxed from important devices and networks are also efective for protecting external systems. To address this, Amazon Web Services is being used to host the research environment – isolated from all personal and public networks and systems, minimising the impact of an attack if it did propagate outside of the system.
			\end{itemize}
		\end{itemize}		
	
	Overall, the ethical concerns around the use of honeypots are widely accepted to be resolved by the nature of the deployment of honeypots: Ultimately, their purpose in deployments are to defend against unethical actions by attackers.
	
	\subsection{Challenges}



\section{Summary}