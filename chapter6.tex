

\chapter{Evaluation}


\section{Experimental Setup}

Describes the experimental setup and the values that were defined for the variables as given in table~\ref{table:experimentsetup}.

\begin{table}[!h]
	\begin{center}
		\begin{tabular}{|c|c|} 
			\hline
			\bf Column 1  & \bf Column 2   \\
			\hline
			Row 1 & Item 1  \\
			Row 2 & Item 1  \\
			Row 3 & Item 1  \\
			Row 4 & Item 1  \\
			\hline
		\end{tabular}
	\end{center}
	\caption[Variables of the experiment]{Caption that explains the table to the reader}	
	\label{table:experimentsetup}
\end{table}


\section{Experiment 1}

Figure~\ref{fig:measurements} shows measurements.

\includewidefigure{measurements}{Measurement of System Wakeups}{Long caption that describes the figure to the reader}{measurements.png}

\section{Experiment 2}

\section{Experiment 3}

\section{Discussion}

It is quite probable that some of the attack patterns may have been different, had a different hosting solution been used. As described by (Vasilokonias? et al), Amazon Web Services have published their IP ranges in the public domain ()https://docs.aws.amazon.com/general/latest/gr/aws-ip-ranges.html)
AWS are very well-known and popular, so are likely an attractive target for this reason.


\section{Summary}